\section{Fase di analisi}
\subsection{Analisi delle caratteristiche dell'utenza}
\label{sub:analisi_delle_caratteristiche_dell_utenza}
\subsubsection{Destinatari}
\label{subs:destinatari}
Il sito \emph{Physisque} si rivolge sia a chi è alle prime armi nel mondo del fitness, sia a chi è un atleta esperto che vuole far parte di una community e condividere i suoi risultati. 
L'interfaccia si propone di essere il più semplice e intuitiva possibile per facilitare tutti gli utenti nella fruizione del sito stesso. I tipi di utenti individuati sono:
\begin{itemize}
    \item l'\textit{utente generico}, che non è registrato al sito o non ha effettuato il login;
    \item l'\textit{utente registrato}, che ha effettuato il login;
    \item l'\textit{utente amministratore}, che possiede privilegi rispetto agli altri utenti.
\end{itemize}

\subsubsection{Funzionalità}
\label{subs:funzionalità}
Qualsiasi tipo di utente può navigare all'interno del sito, visualizzare tutti gli allenamenti descritti, le ricette e le news presenti e prendere visione dei post presenti nel forum.\\
L'utente generico può creare un proprio account tramite la funzionalità di signup. L'utente registrato potrà accedere al proprio account ogni volta che lo desidera. Una volta effettuato l'accesso, l'utente potrà:
\begin{itemize}
    \item visualizzare la propria pagina utente;
    \item modificare i propri dati;
    \item inserire post nel forum o risposte al post di qualcun altro;
    \item lasciare mi piace ai post degli altri utenti.
\end{itemize}
L'utente amministratore, oltre a possedere tutte le funzionalità degli altri utenti, 
può aggiungere una nuova ricetta, 
aggiungere una news, eliminare news, ricette, post o risposte ai post. Ha inoltre la possibilità di promuovere un utente al grado di amministratore, bannare o revocare il ban di un utente ed eliminare l'account di un utente.
