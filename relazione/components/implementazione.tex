\section{Fase di implementazione}
\subsection{Linguaggi}
\subsubsection{HTML}
Il linguaggio di markup scelto per realizzare tutte le pagine del sito è XHTML. Si è preferito questo a HTML5 per i seguenti motivi:
\begin{itemize}
    \item garantisce alta compatibilità con i browser più obsoleti;
    \item i tag <html>, <head>, <title> e <body> sono obbligatori;
	\item gli elementi devono essere nidificati correttamente;
	\item gli elementi devono essere sempre chiusi;
	\item gli elementi devono essere sempre in minuscolo;
	\item i nomi degli attributi devono essere sempre in minuscolo.
\end{itemize}
%non so cosa dire
\subsubsection{CSS}
Per la realizzazione del design del sito è stato adottato CSS3. In ordine di garandire la corretta separazione fra contenuto e presentazione, è stato usato un foglio di stile esterno anziché ricorrere ai tag di stile all'interno del codice XHTML. Come unità di misura si è scelto di utilizzare quelle relative per la loro adattabilità.
%non so cos'altro scrivere
\subsubsection{SQL}
\subsubsection{PHP}
\subsubsection{Javascript}
