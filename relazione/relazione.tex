\input{config.tex}
\title{Progetto di Tecnologie Web}
\author{}
\date{Anno accademico 2020/2021}

\begin{document}
\pagenumbering{gobble}
\maketitle
\begin{figure}[H]
	\centering
	\includegraphics[width=8cm]{img/logo.png}
\end{figure}
\begin{table}[H]
	\centering
	\begin{tabular}{c|c c c}
		\textbf{Componenti} & Kostadinov     & Samuel  & 1187605 \\
		                    & Roveroni       & Emma    & 1169669 \\
		                    & Tesser         & Marco   & 1174009 \\
		                    & Uderzo         & Marco   & 1163774 \\
	\end{tabular}
\end{table}

\begin{center}
	\textbf{Indirizzo sito web}: http://tecweb2021.studenti.math.unipd.it/mtesser\\
	\textbf{Email referente del gruppo}: samuel.kostadinov@studenti.unipd.it
\end{center}

\begin{table}[H]
	\centering
	\begin{tabular}{c|c c}
		\textbf{Utenti} & \textbf{Nickname} & \textbf{Password} \\
		\hline
		Admin           & admin        & admin        \\
		User            & user         & user         \\
		Utente 1        & Mtesser      & MarcoT       \\
		Utente 2        & EmmaRoveroni & EmmaR        \\
		Utente 3        & SamuelK      & SamuelK      \\
	\end{tabular}
\end{table}
\newpage
\pagenumbering{roman}
\tableofcontents
\newpage
\pagenumbering{arabic}
\renewcommand{\abstractname}{Abstract}
\begin{abstract}
	\emph{Physique} è un sito di una palestra che permette a tutti i suoi utenti, siano essi noofiti o esperti, di avere dei consigli legati al mondo del fitness e di interagire tra loro tramite un forum. \newline
	Il sito offre la possibilità di vedere alcuni esempi di allenamenti per tutti, alcune ricette sane che permettono di mantenere la linea e le news riguardanti il mondo del fitness. Permette infine di interagire tra diversi utenti tramite il forum.
\end{abstract}
\newpage
\input{components/analisi}
\newpage
\section{Fase di progettazione}
\subsection{Progettazione}
In una prima fase, il gruppo si è riunito per decidere il tema del sito web da sviluppare. Sono state raccolte le varie idee, decise le funzionalità e gli obiettivi da raggiungere, quali tipologie di argomenti trattare in ciascuna pagina, impostato il layout del sito e decisi i colori da utilizzare. Inoltre sono state decise le tecnologie da usare e si è decisa una prima suddivisione dei compiti individuali. \\
Successivamente è stato progettato il database, nel quale vengono immagazzinate e organizzate le informazioni relative agli utenti e ai contenuti pubblicati sul sito web. Allo stesso tempo, è stato portato avanti lo sviluppo della parte front end del sito, prestando particolare attenzione anche all'accessibilità e alla visualizzazione del sito su dispositivi di diverse dimensioni. \\
Infine, il sito è stato approfonditamente testato al fine di rilevare e risolvere eventuali bug e di validare la compatibilità con i differenti browser e dispositivi.
\subsection{Struttura}
\subsubsection{Header}
L'header contiene il logo, il titolo della pagina su cui ci si trova e due link, uno per il login e l'altro per il signup. Questi ultimi due link sono visibili solo nel caso in cui l'utente non abbia ancora effettuato l'accesso. Dopo che l'utente ha eseguito l'accesso, al posto dei due link login e signup, sarà presente un solo link, ovvero quello per il logout, il quale, una volta cliccato, scollegherà l'utente e lo riporterà alla home page. 
\subsubsection{Breadcrumb}
Lo scopo del breadcrumb è quello di indicare all'utente dove si trova all'interno del sito e il percorso per arrivare in tale pagina, in modo che gli sia più facile orientarsi durante la navigazione all'interno del sito e che possa facilmente tornare indietro. L'ultimo campo, infatti, corrisponde alla pagina corrente e, per evitare link circolari, questo non è cliccabile.
\subsubsection{Container}
La classe container comprende il menù e il contenuto del sito.
\paragraph{Menù}
Il menù è posizionato sulla sinstra e contiene tutte le pagine raggiungibili tramite il sito:
\begin{itemize}
	\item \textit{Home};
	\item \textit{Workout};
	\item \textit{Alimentazione};
	\item \textit{Forum};
	\item \textit{News}.
\end{itemize}
Nel caso in cui l'utente abbia effettuato l'accesso, nel menù sarà presente anche il link alla pagina \textit{Profilo}. Se l'utente è anche amministratore verrà visualizzato anche il link per il \textit{Pannello admin}.
\paragraph{Content}
Lo scopo di content è quello di esporre i contenuti proposti dal sito.
\subparagraph{Home}
La pagina home è quella principale, ovvero la prima che si vede quando si visita il sito. Al suo interno sono presenti una descrizone del sito, le funzionalità che possono essere trovate all'interno del sito ed infine la foto del team.
\subparagraph{Workout}
La pagina workout espone i concetti basilari ed imprescindibili della programmazione dell'allenamento. E' perciò rivolta al principiante che ha intenzione di entrare in palestra per la prima volta, o che si è finora allenato in modo non ottimale.
La pagina è suddivisa in paragrafi che danno un'introduzione al motivo per cui è da preferirsi una programmazione di allenamento studiata.
In fondo alla pagina si possono trovare dei link che rimandano alle principali tipologie di suddivisione dell'allenamento.
\subparagraph{Split}
Le pagine split sono 4:
\begin{itemize}
\item Bro Split;
\item Push-Pull-Legs;
\item Upper-Lower;
\item Full-Body;
\end{itemize}           
Tutte e 4 hanno una struttura simile tra di loro.
Sono suddivise in paragrafi in cui viene spiegato in cosa consiste lo split, per poi elencare i suoi vantaggi e svantaggi ed infine mostrarne un esempio in forma tabellare.
In tutte le pagine è presente un link "Indietro" che riporta alla pagina workout principale. 
\subparagraph{Alimentazione}
La pagina di alimentazione contiene un elenco di ricette, che sono presentate in riquadri contenenti il nome della ricetta, l’immagine
del piatto ed infine il link che apre la pagina completa della ricetta. 
\subparagraph{Ricetta}
La pagina della ricetta completa riporta il titolo della stessa seguito da una foto del risultato finale del piatto. Successivamente vengono elencati gli ingredienti necessari alla preparazione, viene illustrato il procedimento da seguire ed, eventualmente, vengono dati dei consigli per agevolare la riuscita della ricetta.
\subparagraph{Forum}
La pagina del forum, nel caso in cui l'utente non abbia effettuato l'accesso, chiederà all'utente di accedere o registrarsi per lasciare commenti. Nel caso in cui questo non avvenga l'utente avrà solo la possibilità di leggere i commenti lasciati dagli altri utenti e non potrà interagire con il forum.
Al contrario, nel caso in cui l'utente abbia effettuato l'accesso, come prima cosa verrà visualizzato un form per lasciare un commento e successivamente, come nel caso precedente, l'elenco di tutti i post con ora anche la possibilità di rispondere a questi e di lasciare eventualmente un like.
\subparagraph{News}
La pagina delle news riporta, al primo accesso, una lista con tutte le notizie del sito di ogni categoria. Queste, successivamente potranno essere filtrate, e quindi ci sarà la possibilità di visualizzare solo le news relative a: workout, alimentazione o notizie del sito.
\subparagraph{Profilo}
La pagina del profilo permette all'utente di visualizzare e modificare alcune informazioni relative al suo account come: nome, cognome, email e password.
\subparagraph{Pannello admin}
La pagina del pannello admin permette all'amministratore di effettuare delle modifiche,rimozioni od inserimenti all'interno del sito, tra cui: promuovere un utente da standard ad amministratore, eliminare ricette, news, post ed eventualmente relative risposte.
Inoltre potrà anche aggiungere news e ricette, bannare un utente dal forum, per poi eventualmente anche rimuovere il ban, ed infine potrà rimuovere anche degli account.
\subparagraph{Signup e login}
Le pagine di signup e login permettono all'utente rispettivamente di  registrarsi o accedere con le proprie credenziali al sito. Nella pagina signup sono presenti dei campi per inserire username, indirizzo email, nome ,cognome e la password (quest’ultima da inserire due volte per confermare la correttezza). Per il login invece è necessario solamente inserire l'username e la password.
\subsection{Visualizzazione}
Il sito è stato pensato in modo da avere una visualizzazione adatta ad ogni tipologia di dispositivo ed in modo da avere pagine accessibili ad una categoria di utenti più ampia possibile.
Sono state quindi realizzate tre tipologie di visualizzazione del sito: desktop, mobile e stampa.
\paragraph{Desktop}
Nella visualizzazione desktop il layout è diviso in 4 schede.
La prima scheda, ossia l'header,contiene al suo interno: il logo del sito,il nome della pagina in cui ci si trova ed infine le opzioni di autenticazione(login e signup se non si è effettuato l'accesso e logout nel caso contrario).
La seconda scheda, ossia il breadcrumb,contiene il percorso effettuato per arrivare alla pagina corrente.
La terza scheda, ossia il menu,elenca le pagine principali del sito.
La quarta scheda, ossia il content, racchiude il contenuto che si adatta in base alla grandezza della schermata.
\paragraph{Mobile}
Nella visualizzazione mobile è stato implementato il menù adhamburger al fine di gestire al meglio lo spazio disponibile.



  
       
\newpage
\section{Fase di implementazione}
\subsection{Linguaggi}
\subsubsection{HTML}
Il linguaggio di markup scelto per realizzare tutte le pagine del sito è XHTML. Si è preferito questo a HTML5 per i seguenti motivi:
\begin{itemize}
    \item garantisce alta compatibilità con i browser più obsoleti;
    \item i tag <html>, <head>, <title> e <body> sono obbligatori;
	\item gli elementi devono essere nidificati correttamente;
	\item gli elementi devono essere sempre chiusi;
	\item gli elementi devono essere sempre in minuscolo;
	\item i nomi degli attributi devono essere sempre in minuscolo.
\end{itemize}
%non so cosa dire
\subsubsection{CSS}
Per la realizzazione del design del sito è stato adottato CSS3. In ordine di garandire la corretta separazione fra contenuto e presentazione, è stato usato un foglio di stile esterno anziché ricorrere ai tag di stile all'interno del codice XHTML. Come unità di misura si è scelto di utilizzare quelle relative per la loro adattabilità.
%non so cos'altro scrivere
\subsubsection{SQL}
\subsubsection{PHP}
\subsubsection{Javascript}

\newpage
\section{Fase di testing}
Particolare attenzione è stata dedicata alla fase di validazione del sito web realizzato: tutte le pagine sono state sottoposte ad un accorta validazione. Per fare ciò, oltre all'occhio umano, sono stati usati i seguenti strumenti: 
\begin{itemize}
	\item per la validazione del codice XHTML è stato usato il validatore di W3C, Markup Validation Service;
	\item per la validazione dei fogli di stile in CSS è stato usato il validatore di W3C, CSS Validation Service;
	\item per il controllo dei livelli di contrasto di colore presenti nel sito è stata usata l'estensione Wave - web accessibility evaluation tool, e l'estensione del browser Mozilla Firefox NOME
	\item per la simulazione di alcune disabilità visive è stata usata l'estensione Silktide - disability simulator.
\end{itemize}
Inoltre è stata testata la compatibilità del sito su browser più utilizzati. Si garantisce, quindi, il funzionamento del sito per i seguenti browser:
\begin{itemize}
\item Chrome;
\item Mozzila Firefox;
\item Edge (da provare ancora!!!!!);
\item Opera (da provare ancora!!!!);
\item Safari?????
\end{itemize}
\newpage
\input{components/organizzazione}
\end{document}
